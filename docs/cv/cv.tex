\documentclass[11pt,a4paper,sans]{moderncv}
\moderncvstyle{banking}
\moderncvcolor{blue}
\usepackage{etaremune}
\usepackage[utf8]{inputenc}
\usepackage[scale=0.75]{geometry}
\usepackage{import}

% personal data
\name{Ryan}{M. Richard}
\title{Chemistry}
\address{Ames National Labratory and Iowa State University}{Ames, IA}{USA}
\phone[mobile]{Available upon request.}
\email{rrichard@ameslab.gov}
\homepage{https://rmrresearch.github.io/}

%-------------------------------------------------------------------------------
%-- Title
%-------------------------------------------------------------------------------
\begin{document}

\makecvtitle

\small{
	Scientist II at Ames National Laboratory and Adjunct Assistant
    Professor of Chemistry at Iowa State University.  Research interests
    include: \textit{ab initio} methods development, computational chemistry,
    high-performance computing, high-accuracy quantum chemistry, and scientific
	software design.
}

%-------------------------------------------------------------------------------
%-- Education
%-------------------------------------------------------------------------------

\section{Education}
\vspace{5pt}

\begin{itemize}
	\item{\cventry{2008--2013}
		  {Dissertation topic:}
		  {Ph.D. The Ohio State University}
		  {Columbus, OH}
		  {}
		  {\textit{``Increasing the Computational Efficiency of Ab Initio
		  		 Methods with Generalized Many-Body Expansions''}}
 		  {Advisor:  John M. Herbert}
 		  {}}
	\item{\cventry{2004--2008}
		  {Undergraduate: Chemistry major}
		  {B.S. Cleveland State University}
		  {Cleveland, OH}
		  {Advisor: David W. Ball}
		  {}}
\end{itemize}

%-------------------------------------------------------------------------------
%-- Professional Experience
%-------------------------------------------------------------------------------

\section{Professional Experience}
\vspace{5pt}
\begin{itemize}
	\item{\cventry{2018--Present}
		  {Architect and a lead developer of \textsc{NWChemEx}.}
		  {Scientist II, Ames National Laboratory}
		  {Ames, IA}
		  {}
		  {Additional Responsibilities: Mentor undergraduate researchers.}}
	\item{\cventry{2007--2008}
 		  {Research Topics:}
 		  {Research Assistant, NASA Glenn Research Center}
 		  {Cleveland, OH}
 		  {}
 		  {Characterization of degradation properties of ionic liquid based
 		   lubricants for terrestrial and space applications by Raman and
 		   infrared spectroscopies as well as liquid and gas chromatography.}
 	      {}}
\end{itemize}

%-------------------------------------------------------------------------------
%-- Academic Experience
%-------------------------------------------------------------------------------

\section{Academic Experience}
\vspace{5pt}
\begin{itemize}
	\item{\cventry{2022--Present}
	      {High-accuracy benchmark development.}
	      {Adjunct Assistant Professor of Chemistry, Iowa State University}
	      {Ames, IA}
	      {}
	      {Additional Responsibilities: Co-advise graduate students.}}
	\item{\cventry{2017--2018}
		  {Algorithm development, development of massively parallel
		  	 architecture}
		  {Postdoctoral Researcher, Ames National Laboratory}
		  {Ames, IA}
		  {}
		  {Additional Responsibilities: Mentor undergraduate researchers}}
	\item{\cventry{2014--2017}
		  {Algorithm development, high-accuracy bench marking}
		  {Postdoctoral Researcher, Georgia Institute of Technology}
		  {Atlanta, GA}
		  {}
		  {Additional Responsibilities: Mentor undergraduate researchers,
		  	 substitute teach lectures, assist in grant writing}}
	\item{\cventry{2008--2014}
		  {Algorithm development, excited state modeling}
		  {Research Assistant, The Ohio State University}
		  {Columbus, OH}
		  {}
		  {}}
	\item{\cventry{2009--2011}
		  {General chemistry laboratory and recitation, physical chemistry
		   recitation}
		  {Teaching Assistant, The Ohio State University}
		  {Columbus, OH}
		  {}
		  {Additional Responsibilities: Grading, weekly office hours}}
	\item{\cventry{2006--2008}
		{Characterization of new high-energy materials}
		{Research Assistant, Cleveland State University}
		{Cleveland, OH}
		{}
		{}}
\end{itemize}

\section{Software Development Experience}
\begin{itemize}
	\item{\cventry{2017--present}
		{Architect and lead developer}
		{\textsc{NWChemEx}}
		{https://github.com/NWChemEx-Project}
		{}
		{\begin{itemize}
			\item{Package focused on high-performance, massively parallel
			      electronic structure theory.}
		    \item{Contributions: design, plugin framework, CI infrastructure,
		    	  SCF, MP2, MP2-F12, CCSD-F12}
		\end{itemize}}}
	\item{\cventry{2021--present}
          {Founder, architect, project manager, and lead developer}
          {\textsc{GhostFragment}}
          {https://github.com/rmrresearch/GhostFragment}
          {}
          {\begin{itemize}
          		\item{Software for massively parallel generalized many-body
          			  expansion and basis-set superposition corrections.}
                \item{Contributions: concept, design, CI infrastructure,
                	 initial implementations.}
                 \end{itemize}}}
	\item{\cventry{2019--present}
		  {Founder, architect, project manager, and lead developer}
		  {\textsc{CMakePP}}
		  {https://github.com/CMakePP}
		  {}
		  {\begin{itemize}
		  		\item{Object-oriented CMake build system.}
	 		    \item{Contributions: concept, design, CI infrastructure, CMakePP
		   	          language}
 	       \end{itemize}}}
	\item{\cventry{2014--2017}
		  {Developer}
		  {\textsc{Psi4}}
		  {https://psicode.org/}
		  {}
		  {\begin{itemize}
		  		\item{Electronic structure package focused on providing a
		  			  user-friendly experience.}
			  	\item{Contributions: SCF, infrastructure, many-body expansion}
		  	\end{itemize}}}
	\item{\cventry{2008--2013}
		  {Developer}
		  {\textsc{Q-Chem}}
		  {https://www.q-chem.com/}
		  {}
		  {\begin{itemize}
		  		\item{Commercial electronic structure package focused on good
		  			  performance on workstations and small clusters.}
		 	    \item{Contributions: QM/MM}
		   \end{itemize}}}
\end{itemize}

\section{Expertise}
\begin{itemize}
	\item{\textbf{Computational Chemistry Topics:}\\
		  Hartree-Fock, MP2, coupled-cluster, explicit correlation,
		  domain-local methods, fragment-based methods}
	\item{\textbf{Productivity Tools:}\\
		  Git, GitHub, VSCode, Slack}
	\item{\textbf{Computer Languages:}\\
		  C/C++, Python, CMake, \LaTeX, Bash}
	\item{\textbf{Markup Languages:}\\
		  Doxygen, Sphinx-flavored reStructured Text, Markdown}
	\item{\textbf{Python Packages:}\\
		  Jupyter Notebooks, Matplotlib, Numpy, Sphinx}
	\item{\textbf{C++ Libraries:}\\
		  Boost, Catch2, Cereal, PyBind11}
	\item{\textbf{High-Performance Computing Libraries:}\\
		  BLAS, LAPACK, MPI}
\end{itemize}

\section{Honors and Awards}
\vspace{5pt}
\begin{itemize}
	\item{\cventry{2024}
		{Recipient of 1 of 6 BSSw Fellowships}
		{Better Scientific Software Fellow}
		{BSSw.io}
		{}
		{\begin{itemize}
			\item{Recognizes work for promoting better scientific software.}
			\item{Awarded for: Multi-Project CI/CD for modular scientific
				 software}
		\end{itemize}}
	}
	\item{\cventry{2022 -- 2023}
          {Accepted into 2022 -- 2023 cohort}
          {Ames National Laboratory Scientist Leadership Development Program}
          {Ames, IA}
          {}
          {\begin{itemize}
                  \item{Nominees are junior level scientists at the laboratory
                        who show leadership potential.}
                  \item{Program focuses on mentoring and developing leadership
                        skills with a focus on career advancement at a
                        Department of Energy National Laboratory.}
           \end{itemize}}}
    \item{\cventry{2020 -- 2021}
  		 {Accepted into 2020 -- 2021 cohort}
 		 {ISU Research Collaboration Catalysts}
		 {Ames, IA}
		 {}
		 {\begin{itemize}
   		      \item{Nominees demonstrate potential to lead future high-impact
   		      interdisciplinary research teams.}
   		      \item{Provided educational opportunities to refine leadership
   		      skills.}
   	 	  \end{itemize}}}
	\item{\cventry{2016}
		  {252$^{nd}$ American Chemical Society National Meeting and
		  	 Exposition}
	  	 {Finalist for the Emerging Technology in Computational Chemistry
	  	 	Symposium}
	  	 {Philadelphia,PA}
	  	 {}
	  	 {}}
	\item{\cventry{2008--2009}
		  {The Ohio State University}
		  {The Ohio State University Graduate School Fellowship}
		  {Columbus, OH}
		  {}
		  {}}
	\item{\cventry{2007}
		  {Notre Dame College}
		  {Best Undergraduate Presentation}
		  {South Euclid, OH}
		  {}
		  {}}
	\item{\cventry{2004--2008}
          {Cleveland State University}
          {Cleveland State University Honors Program Scholarship}
          {Cleveland, OH}
          {}
          {}}
\end{itemize}

\section{Professional Service}
\vspace{5pt}
\begin{itemize}
	\item{\cventry{2024}
	      {Organizer and moderator.}
	      {Virtual Workshop on Multi-Project CI/CD}
	      {Virtual}
	      {}
	      {Workshop sought to create a community of practice for DevOps 
		  professionals managing CI/CD workflows for multiple software 
		  project.}}
	\item{\cventry{2024}
		  {Member of the Sustainable Scientific Software (S3C) organizing team.}  
	      {National Laboratory Information Technology (NLIT) Summit 24}
		  {Seattle, WA}
		  {}
		  {Helped choose topics for conference and downselect contributed 
		  presentations.}}
	\item{\cventry{2022}
		  {Ames National Laboratory and Iowa State University Exhibit.}
	      {2022 Iowa State Fair}
		  {Des Moines, IA}
		  {}
		  {Explained research at Ames National Laboratory to general public and
		   answered questions.}}
	\item{\cventry{2019--Present}
          {Moderator and Judge}
          {Department of Energy's National Science Bowl}
          {Ames, IA}
          {}
          {Helped setup/teardown and served as moderator and judge.}}
	\item{\cventry{2018}
	      {Session chair}
          {256$^{th}$ National American Chemical Society Meeting.}
          {Boston, MA}
          {}
          {}}
	\item{\cventry{2016}
          {Georgia Institute of Technology}
          {PURA (President's Undergraduate Research Award) application reviewer}
          {Atlanta, GA}
          {}
          {}}
	\item{\cventry{2015}
	     {Session chair}
         {2$^{nd}$ Annual Postdoctoral Research Symposium}
         {Atlanta, GA}
         {}
         {}}
	\item{\cventry{2014}
          {Session chair}
          {1$^{st}$ Annual Postdoctoral Research Symposium}
          {Atlanta, GA}
          {}
          {}}
	\item{\cventry{2014--2017}
        {Co-Organizer}
        {`Postdocs@Tech"}
        {Atlanta, GA}
        {}
        {Organized monthly social events and an annual university research 
		 symposium for postdocs at Georgia Tech.}}
	\item{\cventry{}
          {Journal of Chemical Physics, Journal of Chemical Theory and
          Computation}
          {Peer Reviewer for:}
		  {}
          {}
          {}}
	%\item{\cventry{}
    %      {American Chemical Society}
    %      {Professional Societies}
    %      {}
    %      {}
    %      {}}
\end{itemize}


\section{Active Grants}
\vspace{5pt}
\begin{itemize}
	\item{
		\cventry{2024--2027}
		{Theresa Windus (PI) and \textbf{Ryan M. Richard} (co-PI)}
		{NSF CHE-2348724}
		{Ames, IA}
		{}
		{REU Site: Sutainability Institute for Machine Learning and 
		 Collaborative Open-Source Development of Enzymatic Simulations}
		{Value of award: \$465,000.00}
		{}
	}

	\item{\cventry{2024}
		{\textbf{Ryan M. Richard}\ (PI)}
		{Ames National Laboratory Directed Research and Development
         FY2024-xxxx-01}
		{Ames, IA}
		{}
		{Automating Parameter Selection of Fragment Based Methods for Materials
		Challenges}
		{Value of award: \$287,000.00}
		{}
	}
	\item{
		\cventry{2024--2025}
		{\textbf{Ryan M. Richard}\ (PI)}
		{BSSw Fellowship FY2023-407}
		{Ames, IA}
		{}
		{Multi-Project CI/CD for Modular Scientific Software}
		{Value of award: \$50,000.00}
		{}
	}
\end{itemize}

\section{Completed Grants}
\vspace{5pt}
\begin{itemize}
	\item{
		\cventry{2021--2022}
		{\textbf{Ryan M. Richard}\ (PI)}
		{Ames National Laboratory Directed Research and Development
         FY2022-xxxx-01}
		{Ames, IA}
		{}
		{GhostFragment: Making Strong Electron Correlation Less Scary}
		{Value of award: \$48,000.00}
		{}
	}
	\item{\cventry{2016--2017}
          {\textbf{Ryan M. Richard}\ (PI), C.~D.~Sherrill\ (Co-PI).}
          {Renewal of XSEDE Research Allocation number CHE150006}
          {Atlanta, GA}
          {}
          {Accurate Crystal Lattice Energies Via Pleasantly Parallel Methods}
          {Value of award: \$115,006.00}
          {}}
	\item{\cventry{2015--2016}
          {\textbf{Ryan M. Richard}\ (PI), C.~D.~Sherrill\ (Co-PI).}
          {XSEDE Research Allocation number CHE150006}
          {Atlanta, GA}
          {}
          {Applying Quantum Chemistry to Condensed Phase Systems Using the
		  Many-Body Expansion}
          {}
          {}}
	\item{\cventry{2014--2015}
          {\textbf{Ryan M. Richard}\ (PI), C.~D.~Sherrill\ (Co-PI).}
          {XSEDE Startup Allocation number CHE140145}
          {Atlanta, GA}
          {}
          {Accurate Crystal Properties Via Massively Parallel Quantum Chemistry}
          {}
          {}}
\end{itemize}

\section{Students Mentored}

	\subsection{Graduates}
	\begin{itemize}
		\item{
			\cventry{2023 -- Present}
			{Research focus: Benchmarking fragment-based methods}
			{Heflin, Jacob}
			{Ames, IA}
			{}{}
		}
	\end{itemize}
\subsection{Undergraduates}
\begin{itemize}
	\item{
		\cventry{2023--Present}
		{Research focus: Benchmarking geometry optimizers.}
		{Lewis, John}
		{Ames, IA}
		{}{}
	}
	\item{
		\cventry{2023}
		{Research focus: Implementing capping methods in GhostFragment}
		{Walker Hayes}
		{Ames, IA}
		{}{}
	}
	\item{
		\cventry{2023}
		{Research focus: Implementing capping methods in GhostFragment}
		{Issac Van Orman}
		{Ames, IA}
		{}{}
	}
	\item{
		\cventry{2020--2021}
		{Research focus: Object-oriented CMake.}
		{Theodore Davis}
		{Ames, IA}
		{}{}
	}
	\item{
		\cventry{2020--2021}
		{Research focus: Continuous integration for CMake.}
		{Emin Okic}
		{Ames, IA}
		{}{}
	}
	\item{
		\cventry{2020}
		{Research focus: Object-oriented CMake.}
		{Allison Finger}
		{Ames, IA}
		{}{}
	}
	\item{
		\cventry{2020}
		{Research focus: Object-oriented CMake.}
		{Blake Mulnix}
		{Ames, IA}
		{}{}
	}
	\item{
		\cventry{2020}
		{Research focus: Improving self-consistent field guesses.}
		{Jacob Brunton}
		{Ames, IA}
		{}{}
	}
	\item{
		\cventry{2019--Present}
		{Research focus: CMake documentation and unit testing}
		{Branden Butler}
		{Ames, IA}
		{}{}
	}
	\item{
		\cventry{2018--2019}
		{Research focus: Machine learning molecular structure}
		{Andres Garcia-Alejo}
		{Ames, IA}
		{}{}
	}
	\item{
		\cventry{2017}
		{Research focus: Machine learning applications to quantum chemistry}
		{Brodie Schroeder}
		{Ames, IA}
		{}{}
	}
	\item{
		\cventry{2015--2017}
		{Research focus: Understanding the many-body expansion}
		{Michael Zott}
		{Atlanta, GA}
		{}{}
	}
\end{itemize}


\section{Invited Presentations}
\vspace{5pt}
\begin{etaremune}
    \item[]{}
	\subsection{2024}
	\item{\textbf{Ryan M. Richard}. \textit{PluginPlay: Enabling high-
	      performance scientific software one module at a time}. CECAM Flagship 
		  Workshop. Lausanne, Switzerland. February 2024.}

	\subsection{2023}
	\item{\textbf{Ryan M. Richard}. \textit{NWChemEx: Designing a Computational
	      Chemistry App Store for The Exascale}. APS March Meeting. Las Vegas,
		  NV. March 2023.}
	\item{\textbf{Ryan M. Richard}. \textit{NWChemEx: Challenges Faced in
	      Designing an Electronic Structure Program for the Exascale}. SIAM
		  Conference on Computational Science and Engineering. Amsterdam, New
		  Holland. The Netherlands. February 2023.}

	\subsection{2022}
    \item{\textbf{Ryan M. Richard}. \textit{Software design for the exascale
	      era: The NWChemEx perspective}. 2022 Midwest Regional Meeting for the
		  American Chemical Society. October 2022.}

	\subsection{2019}
	\item{\textbf{Ryan M. Richard}. \textit{The SDE: A General Computational
	      Chemistry Software Framework}. Blue Waters Webinar. March 2019.}
	\item{\textbf{Ryan M. Richard} and Theresa L. Windus. \textit{CMake
	      Packaging Project: Reliable, Reproducible, and Reusable Build Systems
		  Made Easy}. Spring \textsc{NWChemEx} Team Meeting. Richland, WA.
		  March 2019.}
	\item{\textbf{Ryan M. Richard}. \textit{The Simulation Development
	      Environment (SDE): A C++ Framework for Reusable Computational
		  Chemistry}. Society for Industrial and Applied Mathematics Conference
		  on Computational Science and Engineering. Spokane, WA. February 2019.}

	\subsection{2018}
	\item{\textbf{Ryan M. Richard}. \textit{Applying Ab Initio Methods to
	      Large Molecules}. Ames Laboratory.  Ames, IA. August 2018.}

	\subsection{2017}
	\item{\textbf{Ryan M. Richard}. \textit{Exascale Prototyping Via A
	      Computational Chemistry App Store}. 254$^{th}$ National American
		  Chemical Society Meeting.  Washington, D.C. August 2017.}

	\subsection{2016}
	\item{\textbf{Ryan M. Richard}. \textit{Pulsar: A Computational Chemistry 
	      App Store}. \textsc{Psi4} Developers Meeting.  University of Georgia.
		  Athens, GA.  November 2016.}
	\item{\textbf{Ryan M. Richard}. \textit{ForcemanII: Status and Possible 
	      Future Directions}.  \textsc{Q-Chem} Developers Meeting.  University 
		  of Pennsylvania. Philadelphia, PA. August 2016.}
	\item{\textbf{Ryan M. Richard}. \textit{Understanding Chemistry Via Fragment
	      Based Methods}. University of Tennessee.  Knoxville, TN.  February 
		  2016.}
	\item{\textbf{Ryan M. Richard}. \textit{Understanding Chemistry Via Fragment
	      Based Methods}. Youngstown State University.  Youngstown, OH.  
		  February 2016.}
	\item{\textbf{Ryan M. Richard}. \textit{Understanding Chemistry Via Fragment
	      Based Methods}. Tennessee Technological University.  Cookeville, TN.
		  January 2016.}
	
	\subsection{2015}
	\item{\textbf{Ryan M. Richard}. \textit{Understanding Chemistry Via Fragment
	      Based Methods}. University of North Florida.  Jacksonville, FL.  
		  November 2015.}

	\subsection{2008}
	\item{\textbf{Ryan M. Richard}, David W. Ball. \textit{Thermodynamic Studies
	      on the Potential Use of Boron- and/or Nitrogen-Containing Molecules as
		  New High Energy Materials}. Ohio State University. Columbus, OH. 
		  February 2008.}
\end{etaremune}

\section{Contributed Presentations}
\vspace{5pt}
\begin{etaremune}
	\item[]{}
	\subsection{2019}
	\item{\textbf{Ryan M. Richard} and Theresa L. Windus. \textit{CMake 
	      Packaging Project: Reliable, Reproducible, and Reusable Build Systems
		  Made Easy}. 257$^{th}$ National American Chemical Society Meeting. 
		  Orlando, FL. April 2019.}
	\item{\textbf{Ryan M. Richard}, Kristopher Keipert, Thom Dunning Jr., 
	      Robert Harrison, and Theresa L. Windus. \textit{The NWChemEx 
		  Simulation Development Environment - A General Computational Chemistry
		  Framework}. Poster Presentation. Society for Industrial and Applied 
		  Mathematics Conference on Computational Science and Engineering. 
		  Spokane, WA. February 2019.}

	\subsection{2018}
	\item{\textbf{Ryan M. Richard} and Theresa L. Windus. \textit{Leveraging 
	      \textsc{NWChemEx}'s computational chemistry app store to design an 
		  exascale {SCF}}. Oral Presentation. 256$^{th}$ National American 
		  Chemical Society Meeting. Boston, MA. August 2018.}
	\item{\textbf{Ryan~M.~Richard}, Brandon~W.~Bakr, and C.~David~Sherrill. 
	      \textit{Understanding the Many-Body Basis Set Superposition Error: 
		  Beyond Boys and Bernardi}.  Oral Presentation. 50$^{th}$ Midwest 
		  Theoretical Chemistry Conference. Chicago, IL. June 2018.}
	
	\subsection{2017}
	\item{\textbf{Ryan M. Richard}. \textit{Leveraging a Computational Chemistry
	      App-Store for Both Teaching and Researching Chemistry}. Oral 
		  Presentation. 254$^{th}$ National American Chemical Society Meeting.  
		  Washington, D.C. August 2017.}
	
	\subsection{2016}
	\item{\textbf{Ryan M. Richard}, Ben Pritchard, and C.~D.~Sherrill. 
	      \textit{Leveraging a Computational Chemistry App Store to Compute 
		  High Accuracy Lattice Energies of Molecular Crystals}.  Oral 
		  Presentation.  252$^{nd}$ American Chemical Society National Meeting 
		  and Exposition.  Philadelphia, PA. August 2016}
	\item{\textbf{Ryan M. Richard}, Brandon Bakr, and C.~D.~Sherrill.
	      \textit{Understanding Basis Set Superposition Error in Many-Body 
		  Systems: Beyond Boys and Bernardi}.  Oral Presentation.  47$^{th}$  
		  Southeast Theoretical Chemistry Association Annual Meeting.  
		  Tallahassee, FL. May 2016}

	\subsection{2015}
	\item{\textbf{Ryan M. Richard}, Michael S. Marshall, Jean-Luc Br{\' e}das, 
	      Noa Marom, and C.~D.~Sherrill. {\em CCSD(T)/CBS Benchmark Quality 
		  Ionization Potentials (IP) and Electron Affinities (EA).} Poster 
		  Presentation.  2$^{nd}$ Annual Postdoctoral Research Symposium.  
		  Atlanta, GA. October 2015.}
	\item{\textbf{Ryan M. Richard}, C.~D.~Sherrill.  {\em Massively Parallel 
	      Fragment Based Methods as Implemented in \textsc{Psi4}.} Oral 
		  Presentation.  250$^{th}$ American Chemical Society National Meeting
		  and Exposition. Boston, Massachusetts. August 2015.}
	\item{\textbf{Ryan M. Richard}, Michael S. Marshall, Jean-Luc Br{\' e}das, 
	      Noa Marom, and C.~D.~Sherrill. {\em CCSD(T)/CBS Benchmark Quality 
		  Ionization Potentials (IP) and Electron Affinities (EA).} Poster 
		  Presentation.  46$^{th}$ Southeast Theoretical Chemistry Association
		  Annual Meeting.  Orlando, Florida. May 2015.}

	\subsection{2014}
	\item{\textbf{Ryan M. Richard}, C.~D.~Sherrill. {\em Achieving Chemical 
	      Understanding Via High Performance Computing}. Oral Presentation. 
		  1$^{st}$ Annual Postdoctoral Research Symposium.  Atlanta, Georgia.  
		  September 2014.}

	\subsection{2013}
	\item{\textbf{Ryan M. Richard}, J.~M.~Herbert.  {\em Achieving the 
		  Complete-Basis Limit in Large Molecular Clusters: Computationally 
		  Efficient Procedures to Eliminate Basis-Set Superposition Error}.  
		  Oral Presentation.  246$^{th}$ American Chemical Society National 
		  Meeting and Exposition.  Indianapolis, Indiana.  September 2013.}
	\item{\textbf{Ryan M. Richard}, J.~M.~Herbert.  {\em Achieving the 
	      Complete-Basis Limit in Large Molecular Clusters: Computationally 
		  Efficient Procedures to Eliminate Basis-Set Superposition Error}.  
		  Oral Presentation.  68$^{th}$ International Symposium on Molecular 
		  Spectroscopy.  Columbus, Ohio. June 2013.}
	\item{\textbf{Ryan M. Richard}, J.~M.~Herbert. Oral Presentation. 
	      45$^{th}$ Midwest Theoretical Chemistry Conference.  Champaign, 
		  Illinois.  May 2013.}

	\subsection{2012}
		\item{\textbf{Ryan M. Richard}, J.~M.~Herbert. {\em A Generalized 
		      Many-Body Expansion and a Unified View of Fragment-Based Methods 
			  in Electronic Structure Theory}.  Poster presentation.  
			  44$^{th}$ Midwest Theoretical Chemistry Conference.  Madison, 
			  Wisconsin.  June 2012.}

	\subsection{2011}
	\item{\textbf{Ryan M. Richard}, J.~M.~Herbert.  {\em TDDFT Calculations of
	Transient IR Spectra of DNA}.  Oral Presentation.  66$^{th}$ International
	Symposium on Molecular Spectroscopy.  Columbus, Ohio.  June 2011.}
	\item{\textbf{Ryan M. Richard}, J.~M.~Herbert.  {\em Time-Dependent Density-
	      Functional Description of the $^1$L$_a$ State in Polycyclic Aromatic 
		  Hydrocarbons: Charge-Transfer State in Disguise?}.  Poster 
		  Presentation. 43$^{rd}$ Midwest Theoretical Chemistry Conference. 
		  South Bend, Indiana. June 2011.}
	
	\subsection{2010}
	\item{\textbf{Ryan M. Richard}, J.~M.~Herbert.  {\em Determination of 
	      Exciton Length in Aqueous B-DNA Using Long-Range Time-Dependent 
		  Density Functional Theory (LRC-TDDFT)}.  Oral Presentation.  
		  65$^{th}$ International Symposium on Molecular Spectroscopy.  
		  Columbus, Ohio, June 2010.}

	\subsection{2008}
	\item{\textbf{Ryan M. Richard}, D.~W.~Ball.\textit{ Thermodynamic Studies on
		  Boron and Nitrogen Containing Spiropentanes}. Oral Presentation.  
		  American Chemical Society Meeting in Miniature. Oberlin College. 
		  Oberlin, OH.  March 2008.}

	\subsection{2007}
	\item{\textbf{Ryan M. Richard}, D.~W.~Ball. \textit{G2, G3, and Complete 
	      Basis Set Calculations on Small Hydroboranes and Mixed Boron-Nitrogen
		  Containing Rings}. American Chemical Society Meeting in Miniature. 
		  Notre Dame College. South Euclid, OH.  March 2007. Winner of Best 
		  Undergraduate Presentation.}
	\item{\textbf{Ryan M. Richard}, D.~W.~Ball. \textit{New Potential High 
	      Energy Materials}. Cleveland State University Undergraduate Research 
		  Symposium. Summer 2007}
\end{etaremune}

\section{Birds of a Feather Sessions}
\begin{etaremune}
	%\item[]{}
	%\subsection{2024}
	\item{\textbf{Ryan M. Richard}. \emph{Best Practices for Multi-project 
	      CI/CD}. NLIT (National Laboratory Information Technology) Summit 2024. 
		  Seattle, Washington. April 2024.}	
\end{etaremune}

\section{Position Papers}
\begin{etaremune}
	\item{\textbf{Ryan M. Richard}, T.~L.Windus. \emph{Is a Language Barrier
	Impeding Development of Better Scientific Software?}. ASCR (Advanced
	Scientific Computing Research) Workshop on the Science of
	Scientific-Software Development and Use. Virtual. December 13-15 2021.}
\end{etaremune}

\section{Request for Information}
\begin{etaremune}
	\item{\textbf{Ryan M. Richard}, Z.~Crandall, H.~v.~Dam, N.~Govind,
	K.~Kowalski, T.~L.~Windus. Stewardship of Software for Scientific and
	High-Performance Computing. Office of Advanced Scientific Computing
	Research.
	}
\end{etaremune}

\section{Review Articles}
\vspace{5pt}
\begin{etaremune}
	\item{\textbf{Ryan M. Richard}, K.~U.~Lao, J.~M.~Herbert.  {\em Aiming for 
	      Benchmark Accuracy with the Many-Body Expansion}.  Accounts of 
		  Chemical Research.  47 (2014) 2828-2836.}
	\item{L.~D.~Jacobson, \textbf{Ryan M. Richard}, K.~U.~Lao, J.~M.~Herbert.  
	      {\em Efficient Monomer-Based Quantum Chemistry Methods for Molecular 
		  and Ionic Clusters}.  Annual Reports of Computational Chemistry.  
		  9 (2013) 25-56.}
\end{etaremune}

\section{Publications}

\vspace{5pt}
\begin{etaremune}
\item[]{}
	\subsection{2024}
	\item{Z.~Crandall, T.~L.~Windus, \textbf{Ryan M. Richard}. 
	      \textit{CMaize: Simplifying inter-package modularity from the build 
		  up}.Journal of Chemical Physics. 160 (2024) 092502.}

	\subsection{2023}
	\item{V.~Gavini \ldots \textbf{Ryan M Richard}\ldots D.~Perez.
	      \textit{Roadmap on electronic structure codes in the exascale era}.
		  Modelling and Simulation in Materials Science and Engineering. 31
		  (2023) 063301. (30$^{th}$ author of 41)
	}
	\item{\textbf{Ryan M. Richard}, K.~Keipert, J.~Waldrop,
	      M.~Keçeli, D.~Williams-Young, R.~Bair, J.~Boschen,
		  Zachery Crandall, Kevin Gasperich, Quazi Ishtiaque Mahmud,
		  A.~Panyala, E.~Valeev, H.~van Dam, W~A. de Jong,
		  T.~L.~Windus. \textit{PluginPlay: Enabling exascale scientific
		  software one module at a time}. The Journal of Chemical Physics. 158
		  (2023) 184801.
	}

	\subsection{2022}
	\item{B.~Butler and \textbf{Ryan M. Richard}. \textit{CMinx: A CMake
	      Documentation Generator}. Journal of Open Source Software. 7 (2022)
		  4680.
	}

	\subsection{2021}
	\item{E.~Epifanovsky, $\ldots$, \textbf{Ryan M. Richard}, $\ldots$,
	      A.~I.~Krylov. \textit{Software for the frontiers of quantum 
		  chemistry: An overview of developments in the Q-Chem 5 package}. The 
		  Journal of Chemical Physics. 155 (2021) 084801. (139$^{th}$ of 220 
		  authors).}
	\item{Karol Kowalski, $\ldots$, \textbf{Ryan M. Richard}, $\ldots$, Theresa 
	      L. Windus. \textit{From NWChem to NWChemEx: Evolving with the 
		  Computational Chemistry Landscape}. Chemical Reviews. 121 (2021) 4962-
		  4998. (19th of 29 authors).}
	
	\subsection{2020}
	\item{Edoardo Apra, $\ldots$, \textbf{Ryan Richard}, $\ldots$, and Robert 
	      Harrison . \textit{NWChem: Past, Present, and Future}. Journal of 
		  Chemical Physics. 152 (2020) 184102. (82$^{nd}$ of 114 authors).}

	\subsection{2019}
	\item{\textbf{Ryan~M.~Richard}, Colleen Bertoni, Jeffery~S.~Boschen, 
	       Kristopher~Keipert, Benjamin~Pritchard, Edward~F.~Valeev, 
		   Robert~J.~Harrison, Wibe~A.~de~Jong, and Theresa~L.~Windus. 
		   \textit{Developing a Computational Chemistry App Store for the 
		   Exascale Era}. Computing in Science \& Engineering. 21 (2019) 48.}

	\subsection{2018}
	\item{\textbf{Ryan M.~Richard}, Brandon Bakr, and C.~David Sherrill. 
	      \emph{Understanding the Many-Body Basis Superposition Error: Beyond 
		  Boys and Bernardi}. Journal of Chemical Theory and Computation. 14 
		  (2018) 2386.}

	\subsection{2017}
	\item{Robert M. Parrish, $\ldots$,\textbf{Ryan M. Richard}, $\ldots$ 
	      \emph{Psi4 1.1: An Open-Source Electronic Structure Program 
		  Emphasizing Automation, Advanced Libraries, and Interoperability}.  
		  Journal of Chemical Theory and Computation. 13 (2017) 3185. 
		  (10$^{th}$ of 26 authors)}

	\subsection{2016}
	\item{Ka Un Lao , Kuan-Yu Liu ,\textbf{Ryan M. Richard} , and John M. 
		  Herbert. \emph{Understanding the many-body expansion for large 
		  systems. II. Accuracy considerations.}  Journal of Chemical Physics.  
		  144 (2016) 164105.}
	\item{O.~Dolgounitcheva, Manuel~ D{\'i}az–Tinoco, V.~G.~Zakrzewski, 
	      \textbf{Ryan~M.~Richard}, Noa~Marom, C.~David~Sherrill and 
		  J.~V.~Ortiz. \emph{Accurate Ionization Potentials and Electron 
		  Affinities of Acceptor Molecules {IV}: Electron–Propagator Methods}.  
		  Journal of Chemical Theory and Computation. 12 (2016) 627-637.}
	\item{\textbf{Ryan M.~Richard}, Michael~S.~Marshall, Olga~Dolgounitcheva, 
	      J.~V.~Ortiz, Jean-Luc Br{\' e}das, Noa Marom, and C.~David Sherrill.  
		  \emph{Accurate Ionization Potentials and Electron Affinities of 
		  Acceptor Molecules.  {I.} Reference Data at the CCSD(T) Complete Basis
		  Set Limit.}  Journal of Chemical Theory and Computation.  12 (2016) 
		  595-604.}

	\subsection{2015}
	\item{Y.~Shao, $\ldots$, \textbf{Ryan M. Richard},$\ldots$  {\em Advances 
	      in Molecular Quantum Chemistry Contained in the Q-Chem 4 Program 
		  Package}.  Molecular Physics.  113 (2015) 184-215. (22$^{nd}$ author 
		  out of 150).}

	\subsection{2014}
	\item{\textbf{Ryan M. Richard}, K.~U.~Lao, J.~M.~Herbert.  
	      {\em Understanding the Many-Body Expansion for Large Systems.  
		  I. Precision Considerations}.  The Journal of Chemical Physics.  
		  141 (2014) 014108:1-14.}

	\subsection{2013}
	\item{Z.~C.~Holden, \textbf{Ryan M. Richard}, J.~M.~Herbert.  {\em Periodic 
	      Boundary Conditions for QM/MM Calculations: Ewald Summation for 
		  Extended Gaussian Basis Sets}.  The Journal of Chemical Physics.  
		  139 (2013) 244108:1-13.}
	\item{\textbf{Ryan M. Richard}, K.~U.~Lao, J.~M.~Herbert.  {\em Approaching
	      the Complete-Basis Limit with a Truncated Many-Body Expansion}.  The 
		  Journal of Chemical Physics.  139 (2013) 224102:1-11.}
	\item{\textbf{Ryan M. Richard}, K.~U.~Lao, J.~M.~Herbert.  {\em Achieving 
	      the CCSD(T) Basis-Set Limit in Sizable Molecular Clusters: 
		  Counterpoise Corrections for the Many-Body Expansion}.  The Journal of
		  Physical Chemistry Letters.  4 (2013) 2674-2680.}
	\item{\textbf{Ryan M. Richard}, J.~M.~Herbert. {\em Many-Body Expansion with
	      Overlapping Fragments: Analysis of Two Approaches}.  Journal of 
		  Chemical Theory and Computation.  9 (2013) 1408-1416.}

	\subsection{2012}
	\item{\textbf{Ryan M. Richard}, J.~M.~Herbert.  {\em A Generalized Many-Body
	      Expansion and a Unified View of Fragment-Based Methods in Electronic 
		  Structure Theory}. The Journal of Chemical Physics. 137 (2012) 
		  064113.}
	\item{W.~Morales, K.~W.~Street~Jr., \textbf{Ryan M. Richard}, D.~J.~Valco.  
	      {\em Tribological Testing and Thermal Analysis of an Alkyl Sulfate 
		  Series of Ionic Liquids for Use as Aerospace Lubricants}.  Tribology 
		  Transactions.  55 (2012) 815-821.}

	\subsection{2011}
	\item{K.~W.~Street~Jr., W.~Morales, V.~R.~Koch, D.~J.~Valco, 
	      \textbf{Ryan M. Richard}, N.~Hanks.  {\em Evaluation of Vapor Pressure
		  and Ultra-High Vacuum Tribological Properties of Ionic Liquids}. 
		  Tribology Transactions.  54 (2011) 911-919.}
	\item{\textbf{Ryan M. Richard},  J.~M.~Herbert. {\em Time-Dependent Density-
	      Functional Description of the $^1$L$_a$ State in Polycyclic Aromatic 
		  Hydrocarbons: Charge-Transfer Character in Disguise?} Journal of 
		  Chemical Theory and Computation. 7 (2011) 1296-1306. }

	\subsection{2009}
	\item{\textbf{Ryan M. Richard}, D.~W.~Ball. {\em B3LYP Calculations on the 
	      Thermodynamic Properties of a Series of Nitroxycubanes Having the 
		  Formula C$_8$H$_{8-x}$(NO$_3$)$_x$ (x=1--8)}.  Journal of Hazardous 
		  Materials.  164 (2009) 1593-1600.}
	\item{\textbf{Ryan M. Richard}, D.~W.~Ball.  {\em Density Functional 
	      Calculations on the Thermodynamic Properties of a Series of 
		  Nitrosocubanes Having the Formula C$_8$H$_{8-x}$(NO)$_x$ (x=1--8)}.  
		  Journal of Hazardous Materials. 164 (2009) 1552-1555.}

	\subsection{2008}
	\item{\textbf{Ryan M. Richard}, D.~W.~Ball.  {\em  Ab Initio Calculations on
	      the Thermodynamic Properties of Azaspiropentanes}.  The Journal of 
		  Physical Chemistry A.  112 (2008) 2618-2627.}
	\item{\textbf{Ryan M. Richard}, D.~W.~Ball. {\em Enthalpies of Formation of 
	      Nitrobuckminsterfullerenes: Extrapolation to C$_{60}$(NO$_2$)$_{60}$}.  
		  Journal of Molecular Structure: THEOCHEM.  858 (2008) 85-87.}
	\item{\textbf{Ryan M. Richard}, D.~W.~Ball. {\em Ab Initio Calculations on 
	      the Thermodynamic Properties of Spiropentane and its Boron-Containing 
		  Derivatives}.  Journal of Molecular Structure: THEOCHEM.  851 (2008) 
		  284-293.}
	\item{\textbf{Ryan M. Richard}, D.~W.~Ball. {\em Ab Initio Calculations on 
	      the Thermodynamic Properties of Azaborospiropentanes}.  Journal of 
		  Molecular Modeling. 14 (2008) 871-878. }
	\item{\textbf{Ryan M. Richard}, D.~W.~Ball. {\em G2, G3, and Complete Basis 
	      Set Calculations on the Thermodynamic Properties of Triazane}.  
		  Journal of Molecular Modeling.  14 (2008) 29-37.}
	\item{\textbf{Ryan M. Richard}, D.~W.~Ball. {\em G2, G3, and Complete Basis 
	      Set Calculations of the Thermodynamic Properties of Cis- and Trans-
		  Triazene}.  Journal of Molecular Modeling.  14 (2008) 21-27.}
	
	\subsection{2007}
	\item{\textbf{Ryan M. Richard}, D.~W.~Ball. {\em G2, G3, and Complete Basis 
	      Set Calculations of the Thermodynamic Properties of Aminoborane, 
		  Diaminoborane, and Triaminoborane}.  Journal of Molecular Structure: 
		  THEOCHEM.  823 (2007) 6-15.}
	\item{\textbf{Ryan M. Richard}, D.~W.~Ball. {\em B3LYP, G2, G3 and Complete 
	      Basis Set Calculations of the Thermodynamic Properties of Small Cyclic
		  and Chain Hydroboranes}.  Journal of Molecular Structure: THEOCHEM. 
		  814 (2007) 91-98.}
	\item{\textbf{Ryan M. Richard}, D.~W.~Ball. {\em G2, G3, and Complete Basis 
	      Set Calculations of Optimized Geometries, Vibrational Frequencies, and
		  Thermodynamic Properties of Azatriboretidine and Triazaboretidine}.  
		  Journal of Molecular Structure: THEOCHEM.  806 (2007) 165-170.}
	\item{\textbf{Ryan M. Richard}, D.~W.~Ball.  {\em Optimized Geometries, 
	      Vibrational Frequencies, and Thermochemical Properties of Mixed Boron- 
		  and Nitrogen-Containing Three-Membered Rings}. Journal of Molecular 
		  Structure: THEOCHEM.  806 (2007) 113-120.}

	\subsection{2006}
	\item{\textbf{Ryan M. Richard}, D.~W.~Ball. {\em G2, G3, and Complete Basis 
	      Set Calculations of the Thermodynamic Properties of Boron-Containing 
		  Rings: cyclo-CH$_2$BHNH, 1,2-, and 1,3-cyclo-C$_2$H$_4$BHNH}. Journal 
		  of Molecular Structure: THEOCHEM.  776 (2006) 89-96. }
\end{etaremune}
\end{document}


%% end of file `template.tex'.
